%%%%%%%%%%%
%% Home work template for Graduate School
%% Author : Thamme Gowda N.
%% Originally from  https://github.com/thammegowda/hw-tex-templ
%%%%%%%%%%%%%%

\documentclass[letterpaper,doc,notimes]{apa6}
%Required by APA6 package
\usepackage[normalem]{ulem}
\usepackage[english]{babel}
\usepackage[utf8x]{inputenc}
\usepackage{amsmath}
\usepackage{graphicx}
\usepackage{adjustbox}

%Oft-used, oft-abused
\usepackage{afterpage}
\usepackage{booktabs}
\usepackage{caption}
\usepackage{censor}
\usepackage{color}
\usepackage{csquotes}
\usepackage{enumitem}
\usepackage{float}
\usepackage{hyperref}
\usepackage{lmodern}
%\usepackage{media9}
\usepackage{multirow}
\usepackage{outlines}
\usepackage{pdfpages}
\usepackage{placeins}
\usepackage{soul}
\usepackage{tabularx}
\usepackage[colorinlistoftodos]{todonotes}
\usepackage{xcolor}


\setenumerate[1]{label=\Roman*.}
\setenumerate[2]{label=\Alph*.}
\setenumerate[3]{label=\roman*.}
\setenumerate[4]{label=\alph*.}

\title{ \textbf{ USC CSCI 544 HOMEWORK 1 REPORT} }
\shorttitle{USC CSCI 544 HW 1}
\author{\textsc{ThammeGowda Narayanaswamy}}
\affiliation{ tnarayan@usc.edu \\ ID : 2074-6694-39 \\ Department of Computer Science \\ Viterbi School of Engineering \\ University of Southern California \\ Los Angeles, CA }

%\note{September 13, 2016}
\note{\today}
\authornote{Produced for Fall 2016 section of CSCI 544, ``Applied Natural Language Processing'', taught by Dr. Mark Core and Dr. Kallirroi Georgila at the University of Southern California}


\begin{document}

\maketitle
\newpage

\section{1. 100\% of Training data }

1. Performance on the development data with 100\% of the training data

1a. spam precision:

1b. spam recall: 

1c. spam F1 score: 

1d. ham precision:

1e. ham recall: 

1f. ham F1 score:


\section{2. 10\% of Training data}

2. Performance on the development data with 10\% of the training data

2a. spam precision:

2b. spam recall: 

2c. spam F1 score: 

2d. ham precision:

2e. ham recall: 

2f. ham F1 score:


\section{3. Enhancements tried}

3. Description of enhancement(s) you tried (e.g., different approach(es) to smoothing, treating common words differently, dealing with unknown words differently):


\section{4. Best Performance}

4. Best performance results based on enhancements. Note that these could be the same or worse than the standard implementation.

4a. spam precision:

4b. spam recall: 

4c. spam F1 score: 

4d. ham precision:

4e. ham recall: 

4f. ham F1 score:

\end{document}